\documentclass{jsarticle}
\usepackage{amsmath}
\usepackage{comment}

\title{ソフトウェア1課題(lec5 宿題1,2)}
\author{340481H 電子情報工学科内定 中里徳彦}
\date{2014/11/12}
\begin{document}

\maketitle

\section{宿題1:int a[30]; という宣言と,int *a[30]; という宣言の違いについて,自分の言葉で説明しなさい。必要に応じて図を使うとよい。}

int a[30]はint型の変数を30個持つ配列aをメモリに確保するという意味である。一方int *a[30]はint型の変数を指し示す(予定の)ポインタが30個ある配列aをメモリに確保するという意味である。前文に(予定の)とあるのは、int *a[30]の宣言を行なった時点ではどの変数を指し示すのかが決まっておらず、まだ何も指し示していないからである。

\section{宿題2:ポインタのポインタ int **p; について,自分の言葉で説明しなさい。必要に応じて図を使うとよい。}

ポインタのポインタint **p;はint型の数値を格納する変数を指し示すポインタを指し示すポインタである。int型の数値を格納する変数をs, その変数を指し示すポインタをint *rとすると、ポインタpにはポインタrのアドレスが、*pとしてポインタの指し示す先の数値を見るとポインタrの数値、つまりsのアドレスが見れる。sの数値を見たい場合には、**pとしなければならない。これは宣言int **p;と表現がかぶっているが、宣言はintを指し示すポインタを指すポインタ型の宣言であるのに対し、宣言した後に使われる**pはポインタpが指し示す先のポインタが指し示す先の数値という意味であり、全く違う意味である。
\end{document}
