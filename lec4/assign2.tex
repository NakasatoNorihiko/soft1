340481H 電子情報工学科 中里徳彦

ソフトウェア1 lec4 宿題2

Nビットのときのnの2の補数とは、2^N - nのことである。例えば4ビットでn=3の補数は2^4 - 0011(2) = 1101(2) (十進法で13)である。コンピュータでは-n(n>0)をnの2の補数として表現することが多い。つまり-3は1101(2)として表現される。nの2の補数はビットを反転し+1することで求めることができる。
このように定義することによって、負の数の和を求める際に、正の数同士の和を求めるのと同じ方法を使うことができる。Nビットにおいて任意のN,M(n,m > 0, n,m < 2^(N-1), n,mは整数)で、n+(-m)を計算することを考える。-m = 2^N - mとして表現されている。この表現を用いて-mを正の数と考えて計算すると、n+(-m) = n + 2^N -m = (n-m) + 2^Nである。今、Nビットを考えているので2^Nはオーバーフローし考えなくてよいから、2の補数表現を用いて正の数同士と同じように和を求めても正しい答えが求まることがわかる。このように負の数の加算においても正の数の加算と同じ方法を用いることができるのが負の数に2の補数表現を用いた時の利点である。
